\documentclass[a4paper]{article}

\usepackage{amsmath}
\usepackage{amssymb}
\usepackage{parskip}
\usepackage{fullpage}
\usepackage{geometry}
\usepackage{hyperref}
\usepackage{url}
\usepackage{tikz}
\usetikzlibrary{lindenmayersystems}

\title{%
    Lindenmayer's Garden \\
    \phantom{} \\
    \large Scuola d'Arti e Mestieri di Trevano (SAMT) \\
    \large Abstract
}

\author{Paolo Bettelini}

\date{}

\geometry{
    a4paper,
    bottom=3cm,
    headheight=4cm,
    top=2cm
}

\hypersetup{
    colorlinks=true,
    linkcolor=black,
    urlcolor=blue,
    pdftitle={Lindenmayer's Garden - Abstract},
    pdfpagemode=FullScreen,
}

\begin{document}

\pgfdeclarelindenmayersystem{A}{
  \symbol{S}{\pgflsystemstep=0.625\pgflsystemstep}
  \rule{X->S[-FX]+FX}
}

\begin{minipage}{0.7\textwidth}
    \maketitle
\end{minipage}
\begin{minipage}{0.3\textwidth}
    \begin{tikzpicture}[rotate=90]
        \draw
            [green!50!black,thin,line cap=round]
            l-system [l-system={A,axiom=FX
            ,order=10,angle=40,step=1.75cm}];
    \end{tikzpicture}
\end{minipage}

\vspace{1.5cm}


\begin{figure}[h]
    \begin{minipage}{0.5\textwidth}
    \end{minipage}
    \begin{minipage}{0.5\textwidth}
    \end{minipage}
\end{figure}

\begin{minipage}{0.5\textwidth}
\begin{itemize}
    \item \textbf{Section}: Computer Science
    \item \textbf{Year:} Fourth
    \item \textbf{Class:} LPI
    \item \textbf{Supervisor:} Geo Petrini
\end{itemize}
\end{minipage}
\begin{minipage}{0.5\textwidth}
\begin{itemize}
    \item \textbf{Expert:} Gionata Genazzi
    \item \textbf{Title:} Lindenmayer's Garden
    \item \textbf{Timeline}: 2023-05-02 - 2023-05-26
    \item \textbf{Presentation:} 2023-06-01 14:00
\end{itemize}
\end{minipage}

\vspace{2cm}

\thispagestyle{empty} % no page number

\section*{Abstract}

Lindenmayer Systems, or commonly L-systems,
are formal grammars used for generating complex patterns and structures.
They were introduced in 1968 by Aristid Lindenmayer, a Hungarian theoretical biologist and botanist.
These systems are exceptionally good at representing natural growth of trees, algae, bushes and such.
L-systems can also be used to draw any sort of fractal, with a variable level of detail.
The goal of this project is to stimulate creativity
by creating a L-system playground.
The program provides a sophisticated deterministic context-free grammar
with advanced drawing features, stochastic behavior and an animation system. \\
Furthermore, L-systems are Turing-compete, meaning that anything can be computed
with them.
Any geometrical shape can be mathematically modelled with these systems. \\
There are many programs that can be used to render L-systems, but none of them
approaches the topic the same way this project is intended to.
The philosophy behind this implementation is to let the user experiment
with arbitrary mathematical expressions, and be as flexible as possible.

\section*{Execution}

This project has been built using GTK4 for the desktop application
and Cairo for the L-systems rendering.
The code was written using the Rust programming language,
which is known for his memory safety and high performance.
A UTF-8 file format (\texttt{.lsys}) was developed to represent
L-systems.


\section*{Results}

The objective of the project has been successfully achieved, and all the requirements have been met.
The final product is user-friendly and allows the user to draw L-systems with ease.
The application is able to import \texttt{.lsys} files,
edit them in real time using the graphical user interface and export them.
The topic of the project was one of the best ones I had ever worked on.
I really enjoyed implementing new features and I spent hours playing
with my own product.

\end{document}